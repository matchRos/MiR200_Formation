Attach two Gazebo models with a virtual joint in a generalized grasp hack

\section*{Build}


\begin{DoxyCode}
1 mkdir -p gazebo\_link\_attacher\_ws/src
2 cd gazebo\_link\_attacher\_ws/src
3 catkin\_init\_workspace
4 git clone https://github.com/pal-robotics/gazebo\_ros\_link\_attacher.git
5 cd ..
6 catkin\_make
7 source devel/setup.bash
\end{DoxyCode}


\section*{Launch}

\begin{DoxyVerb}roslaunch gazebo_ros_link_attacher test_attacher.launch
\end{DoxyVerb}


Empty world with the plugin {\ttfamily libgazebo\+\_\+ros\+\_\+link\+\_\+attacher.\+so} loaded (in the {\itshape worlds} folder).

Which provides the {\ttfamily /link\+\_\+attacher\+\_\+node/attach} service to specify two models and their links to be attached.

And {\ttfamily /link\+\_\+attacher\+\_\+node/detach} service to specify two models and their links to be detached.



\section*{Run demo}

In another shell, be sure to do {\ttfamily source devel/setup.\+bash} of your workspace. \begin{DoxyVerb}rosrun gazebo_ros_link_attacher spawn.py
\end{DoxyVerb}


Three cubes will be spawned. \begin{DoxyVerb}rosrun gazebo_ros_link_attacher attach.py
\end{DoxyVerb}


The cubes will be attached all between themselves as (1,2), (2,3), (3,1). You can move them with the G\+UI and you\textquotesingle{}ll see they will move together. \begin{DoxyVerb}rosrun gazebo_ros_link_attacher detach.py
\end{DoxyVerb}


The cubes will be detached and you can move them separately again.

You can also spawn items with the G\+UI and run a rosservice call\+: 
\begin{DoxyCode}
1 rosservice call /link\_attacher\_node/attach "model\_name\_1: 'unit\_box\_1'
2 link\_name\_1: 'link'
3 model\_name\_2: 'unit\_sphere\_1'
4 link\_name\_2: 'link'"
\end{DoxyCode}


And same thing to detach\+: 
\begin{DoxyCode}
1 rosservice call /link\_attacher\_node/detach "model\_name\_1: 'unit\_box\_1'
2 link\_name\_1: 'link'
3 model\_name\_2: 'unit\_sphere\_1'
4 link\_name\_2: 'link'"
\end{DoxyCode}


\section*{Current status}

It works!

$\sim$$\sim$\+Currently it crashes with\+:$\sim$$\sim$


\begin{DoxyCode}
1 ***** Internal Program Error - assertion (self->inertial != \_\_null) failed in static void
       gazebo::physics::ODELink::MoveCallback(dBodyID):
2 /tmp/buildd/gazebo4-4.1.3/gazebo/physics/ode/ODELink.cc(178): Inertial pointer is NULL
3 Aborted (core dumped)
\end{DoxyCode}


$\sim$$\sim$\+Which I\textquotesingle{}ve only seen this other useful information\+: \href{https://bitbucket.org/osrf/gazebo/issues/1177/removing-moving-model-with-ode-friction}{\tt Bitbucket gazebo removing moving model with ode friction fails}. But it didn\textquotesingle{}t help me solve my crash. I guess when attaching one model to the other it removes the second one to re-\/create it attached to the first or something like that.$\sim$$\sim$

$\sim$$\sim$\+And also this other issue\+: \href{https://bitbucket.org/osrf/gazebo/issues/1077/visualizing-dynamically-created-joints}{\tt Visualizing dynamically created joints} made me add a couple of lines more.$\sim$$\sim$

$\sim$$\sim$\+The method to attach the links is based on the grasp hack of the Gripper in gazebo/physics\+: \href{https://bitbucket.org/osrf/gazebo/src/1d1e3a542af81670f43a120e1df7190592bc4c0f/gazebo/physics/Gripper.hh?at=default&fileviewer=file-view-default}{\tt Gripper.\+hh} \href{https://bitbucket.org/osrf/gazebo/src/1d1e3a542af81670f43a120e1df7190592bc4c0f/gazebo/physics/Gripper.cc?at=default&fileviewer=file-view-default}{\tt Gripper.\+cc}$\sim$$\sim$

$\sim$$\sim$\+Which is to create a revolute joint on the first model (with range of motion 0) linking a link on the first model and a link on the second model.$\sim$$\sim$ 